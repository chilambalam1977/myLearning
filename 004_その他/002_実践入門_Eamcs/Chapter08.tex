%%
%% 章:最新情報を入手するには
%%------------------------------------------------------------------------------------------------------------------------------%%
\chapter{最新情報を入手するには}
%%
%% 節:開発状況
%%--------------------------------------------------------------------------------------------------------------------%%
\section{開発状況}
Emacs は GNU プロジェクト\footnote{https://www.gnu.org/software/emacs}というフリーウェア開発の総本山で開発されている。
%%
%% 項:フリーソフトウェアとしての Emacs
%%----------------------------------------------------------------------------------------------------------%%
\subsection{フリーソフトウェアとしての Emacs}
ここで言うフリーソフトウェアとは当然ながら価格が無料という意味ではない。
また、オープンソースソフトウェアとも異なる。
Richard M.Stallman 氏を中心とするフリーソフトウェア運動の歴史は、自由を守るための戦いであり、全てのソフトウェア開発者が多かれ少なかれ恩恵を享受している。\\

従って、その賛否に関わらずフリーソフトウェアの思想に一度は触れておくべきだと考える。
開発者の生活において無くてはならないフリーソフトウェアが、どのような歴史的経緯で誕生したのかを知ることは決して無駄にはならないはずである。\\

このような話に興味を持った者は「GNUプロジェクトの理念」\footnote{https://www.gnu.org/philosophy/philosophy.ja.html}や「GNU宣言」\footnote{https://www.gnu.org/gnu/manifesto.html}または Mikiya Okuno 氏の「なぜリチャード・ストールマンはオープンソースを支持しないか」\footnote{https://nippondanji.blogspot.jp/2011/06/blog-post\_17.html}などの記事を参照して、自分自身の頭で咀嚼してみるとよいだろう。
とにかく、Emacs は Linux カーネルと並んで巨大で歴史あるフリーソフトウェアプロジェクトであり、誰でもソースコードを入手して開発に参加することができる。
Emacs のメインコードは C 言語で書かれているが、Emacs のソースコードは C 言語で書かれたプログラムの中でも綺麗だと言われている。
C 言語プログラミングに興味のある者は、一度コードリーディングをしてみるのもいいだろう。
%%
%% 項:ソースコードリポジトリ
%%----------------------------------------------------------------------------------------------------------%%
\subsection{ソースコードリポジトリ}
Emacs のコードリポジトリは Savannah\footnote{https://savannah.gnu.org} という開発ホスティングサイトで管理されている\footnote{https://savannah.gnu.org/projects/emacs}。
40年以上の歴史を持つ Emacs はバージョン管理システムも RCS(\emph{Revision Control System})、CVS(\emph{Concurrent Versions System})、Bazaar、そして Git と移り変わってきたが、Eric S.Raymond 氏の尽力により過去の記録をそのまま現在にまで引き継いでいる。
%%
%% 款:Git を利用してソースコードを取得する
%%------------------------------------------------------------------------------------------------%%
\subsubsection{Git を利用してソースコードを取得する}
Git が利用可能な環境であれば、プロジェクトディレクトリから \texttt{git clone git://git.sv.gnu.org/emacs.git} でソースコードを取得することが可能である。
%%
%% 節:開発版の試用
%%--------------------------------------------------------------------------------------------------------------------%%
\section{開発版の試用}
ソフトウェアの開発はリリース版(安定版)と開発版に分かれていることが多く、Emacs にもリリース版と開発版が存在する。
一般的なソフトウェアはリリースまでにアルファ版、ベータ版を経て、RC(\emph{Release Candidate})版で問題がなければ安定版としてリリースすることが多い。
Emacs の場合はこういった区切りとなるリリースは行われず、ある日を境にプレテスト版という形でテストリリースが行われ、この時点で新規機能の追加は終了してバグフィックスを繰り返して安定版がリリースされる流れとなっている。
%%
%% 項:開発版のビルド
%%----------------------------------------------------------------------------------------------------------%%
\subsection{開発版のビルド}
開発版の Emacs とは、基本的にはソースコードをチェックアウトしてから自らビルドしたものを指す。
一昔前では、自身でソフトウェアをビルドするためには様々な知識が必要であったが、最近では要領さえ理解すれば誰でも簡単にビルドすることができる。
開発版の Emacs を利用すると次に搭載される新機能などをいち早く試すことができるので、興味のあるものは是非ビルドして利用してみるとよいだろう。
%%
%% 款:README を読む
%%------------------------------------------------------------------------------------------------%%
\subsubsection{README を読む}
ソフトウェアをビルドするための情報は、一般的にソースコードに同梱されている README ファイルに詳しく記載されている。
何か問題が生じた場合は、Web で検索する前にまず README ファイルなどオフィシャルのドキュメントを読む癖を付けるべきである。\\

Emacs の README は各種ディレクトリとファイルの説明が記載されている。
これによると、インストールに関する説明は INSTALL ファイルに記述されていることが分かる。
但し、Git からソースコードを取得してインストールする場合は INSTALL ファイルだけではなく INSTALL.REPO ファイルにも目を通しておく必要があるので注意が必要である。
%%
%% 款:configure ファイルを作成する
%%------------------------------------------------------------------------------------------------%%
\subsubsection{configure ファイルを作成する}
Git リポジトリから取得したソースコードには configure ファイルが含まれていない場合がある。
その場合、autogen.sh スクリプトを実行して作成する必要がある。
これには autoconf~2.65 以上のバージョンが要求される(2017年10月現在)。
無事に configure スクリプトが作成することができたら、後は第 2 章で解説した通りの方法でビルドすればよい。
%%
%% 節:新しい情報に触れるには
%%--------------------------------------------------------------------------------------------------------------------%%
\section{新しい情報に触れるには}
もし好きな物事に対して人よりも一歩でも先に進みたければ、敏感にアンテナを張り巡らせて、重要であり、ホットであり、そして正しい情報をいち早くキャッチしなければならない。
但し、闇雲に Web 上に溢れる情報を漁るばかりでは情報の波に飲み込まれてるばかりで、なかなか先に先に進むことができないだろう。\\

人よりも敏感にアンテナを張るためには、より質の高い情報が集まる場所を見つける必要がある。
質の高い情報が集まる場所には必ずレベルの高い者たちが集まっており、そこで活動することによって最前線で活動することが可能となる。
本節ではそのような場所を紹介していく。
%%
%% 項:メーリングリスト
%%----------------------------------------------------------------------------------------------------------%%
\subsection{メーリングリスト}
Emacs 開発の議論は、主にメーリングリスト\footnote{http://savannah.gnu.org/mail/?group=emacs}の emacs-devel で行われている。
現在、Emacs の開発がどのように進んでいるかをチェックしたいと思ったら、メーリングリストを読んでみるとよい。
また、メーリングリストへのポストはアーカイブ\footnote{http://lists.gnu.org/archive/html/emacs-devel}が用意されているので、メーリングリストに参加しなくても確認可能である。
%%
%% 項:EmacsWiki:世界最大の Emacs コミュニティ
%%----------------------------------------------------------------------------------------------------------%%
\subsection{EmacsWiki:世界最大の Emacs コミュニティ}
メーリングリストに次いで Emacs 開発者のコミュニティとして規模の大きなものは EmacsWiki\footnote{https://www.emacswiki.org} である。
本稿でも数々の拡張機能を EmacsWiki から紹介してきた。
EmacsWiki は Emacs 開発者にとってノウハウ共有の場であり、作品を公開する場ともなっている。
EmacsWiki の RecentChanges\footnote{https://www.emacswiki.org/emacs/RecentChanges} の RSS リーダに登録し、更新をチェックしておくと世界の誰かが作成した拡張機能をいち早く知ることができる。更新頻度は多くても 1 日 10 ページ程度となっており、読むのに疲れることもない。
%%
%% 項:メーリングリスト、Wiki、その先へ
%%----------------------------------------------------------------------------------------------------------%%
\subsection{メーリングリスト、Wiki、その先へ}
メーリングリストや EmacsWiki はオフィシャルな情報が集まる場所だが、実際にはそこに集まる情報だけでなく、世界中の開発者が日夜 Emacs に関する Tips を公開して拡張機能を開発している。
それらの多くは開発者個人のブログや Web サイトで公開されることが多い。\\

Twitter などで開発者同士のコミュニティを楽しむのもよいが、やはりまとまった情報はブログを読むのが一番であろう。
開発者たちのブログを紹介してもよいのだが、国内外を問わず非常に多く存在するため、ここでは紹介しきることはできない。
そこで、簡単に見つける方法を一部紹介しておく。\\

GitHub には言語別の検索機能やトレンドを見つける機能が備わっており、Elisp のリポジトリを検索したり、活発なリポジトリを調べることができる。
ここから自身にとって有益な拡張機能を作成している作者を見つけて、その作者をアクティビティをチェックするというのがその方法である。\\

また、はてなブックマークのタグ「emacs」を含む新着エントリの RSS をチェックするのもよいだろう。
日本人向けの情報としては EmacsJP と Slack チャット\footnote{http://emacs-jp.slack.com}も存在するので、興味のあるものは參加してみるとよいだろう。
