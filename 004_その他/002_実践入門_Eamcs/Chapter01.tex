%%
%% 章:Emacs の世界へようこそ
%%------------------------------------------------------------------------------------------------------------------------------%%
\chapter{Emacs の世界へようこそ}
%%
%% 節:多くの人に愛される歴史あるエディタ
%%--------------------------------------------------------------------------------------------------------------------%%
\section{多くの人に愛される歴史あるエディタ}
Emacs はテキストエディタである。
プログラムを書くためのツールであると思われがちだが、実はそうではない。
Emacs の持つポテンシャルをうまく言い表す言葉として「Emacs は環境(OS)である」\footnote{http://c2.com/cgi/wiki?EmacsAsOperatingSystem}という表現が存在する。
%%
%% 項:ユーザが自由に機能を変更・追加(拡張)することができる
%%----------------------------------------------------------------------------------------------------------%%
\subsection{ユーザが自由に機能を変更・追加(拡張)することができる}
Emacs ではテトリスをプレイすることができる。
電卓も利用できる。
シェル\footnote{UNIX 系 OS においてコマンドによって OS を操作するためのプログラムのこと}も装備している。
メールも送受信できるし、ブラウザとしても使える。
ゲームからインターネットまで何でもできる、そういったところが「環境」と呼ばれる所以である。\\

Emacs が他のエディタと最も異なる点は、Emacs Lisp(以下 Elisp と記す)という誰でも気軽に触れることのできる独自のプログラミング言語によって、そのほとんどの機能が作成されている点である。
上記の様々な機能も全て、この Elisp によって作成されている。
そのため、ユーザの誰もが好きなように機能を変更・追加(拡張)することが可能となっている。
故に、Emacs は「Programmable Editor(プログラム可能エディタ)」とも呼ばれる。
%%
%% 項:使いこなせば強力な味方になる
%%----------------------------------------------------------------------------------------------------------%%
\subsection{使いこなせば強力な味方になる}
本稿を閲覧している者は、大きく 2 つのグループに分けられるのではないかと考えている。
一方は、まだ Emacs を利用したことはないが、これから利用してみたいと考えている者。
もう一方は、Emacs を既に利用しているが、もっと使いこなしたいと考えている者である。\\

本稿は、その両者を対象として想定している。
今まで Emacs に触れたことのない者にとっては、Emacs を活用することができるようになる手助けとなるよう基礎的な利用方法から解説していく。
既に利用している者にとっても、最新の Emacs についての話題は勿論、今までコピーで済ませていた設定の意味を正しく理解し、自分の思い通りに設定することができるようになるための足掛かりとなるよう解説していく。\\

黒い画面、コマンド、キーボードショートカットなどは初学者にとって不安と戸惑いを与えるだろう。
しかし、これらはハッカーに強い力を与えてくれる。
映画「ソーシャルネットワーク」では、冒頭で主人公の Mark Zuckerberg 氏が軽やかな手つきで Emacs を操り、瞬く間に Web サービスを構築していく姿が描かれている。
そういったハッカーと呼ばれる達人たちは、頭から勢い良く溢れ出すコードを即座にプログラムに変えていくのである。\\

Emacs は古典的でありながらも信じられないほどパワフルな編集機能を与えてくれる。
使いこなすことができれば、これほど力強い味方はないだろう。
%%
%% 項:本当は難しくない Emacs
%%--------------------------------------------------------------------------------------------------------------------%%
\subsection{本当は難しくない Emacs}
よく「Emacs の操作は難しい」と言われる。
しかし、それは誤解である。
タッチタイピングも 1 日で身に付くものではないが、特別な訓練が必要であるわけではなく、ただキーボードをタイプし続けていれば自然に身に付くものである。
それと同様に、Emacs も何気なく使い続けていると自然と指が慣れてきて、気付けばごく普通に使いこなせるようになるのである。\\

しかし、自在に操作ができるようになっただけでは、まだ Emacs の本当の実力を引き出せてはいない。
柔軟な設定方法を学び、気に入らない部分があれば自分の手に馴染むようにカスタマイズする、そして世界に 1 つだけの自分自身の Emacs を構築することができるようになって初めて一人前となるのである。
更に、過去の Emacs とは違い、今日では使いこなせるようになるための近道となるような便利なツールが多数存在する。
それらの存在を知ることで、より習得が楽になるだろう。
%%
%% 節:本当のエディタにできること
%%--------------------------------------------------------------------------------------------------------------------%%
\section{本当のエディタにできること}
Emacs は非常に多機能である。
しかし、応用的な機能を除いて、エディタとして本当に必要な機能にはどのようなものがあるのだろうか?
%%
%% 項:エディタが本来持つべき機能
%%----------------------------------------------------------------------------------------------------------%%
\subsection{エディタが本来持つべき機能}
テキストデータを編集する際、最もエディタの恩恵を受けていると感じるのは、編集したいと思った箇所にピンポイントで移動することができたり、繰り返しとなる作業を自動化し本人の気付いていないミスを防ぐことだろう。
これらを実現するために、エディタとして最低限必要となる機能には次のようなものが挙げられる。
\vspc{-0.50zw}\begin{itemize}\setlength{\leftskip}{-1.00zw}%\setlength{\labelsep}{+1.00zw}
\item \textgt{検索と置換}             \\
  検索によって、巨大なファイルであっても即座に目的の箇所にカーソルを移動することができる。また、置換により語彙の一括変換や煩雑な入力を手助けしてくれる。
\item \textgt{入力補助・補完}         \\
  頻繁に入力する定型文、コーディング規約、関数、変数名など、予め定められた入力があれば、それを僅かなタイプで入力できるよう適切に補助してくれる。
\item \textgt{シンタックスハイライト} \\
  コードの構文(シンタックス)に従って文字を装飾することで、表示を見やすくしたり、入力ミスを視覚的に示唆してくれる。
\item \textgt{インデント}             \\
  インデントを整えることでコーディング規約を守り、コードが読み易くなる。自分好みのインデント幅に一括で変更することなども可能である。
\end{itemize}\vspc{-0.50zw}
これらの機能はタイプミスの削減、可読性や入力速度の向上などに大きく貢献し、テキストエディタには欠かせない重要な機能である。
しかし、これらの機能は大抵のエディタに搭載されている機能である。
そこで、Emacs ならではの便利機能をいくつか挙げてみる。
\vspc{-0.50zw}\begin{itemize}\setlength{\leftskip}{-1.00zw}%\setlength{\labelsep}{+1.00zw}
\item \textgt{文法チェック}           \\
  プログラム実行(コンパイル)時のエラーなどを予め検出することで、予期せぬエラーを未然に防ぐことができる。
\item \textgt{矩形編集}               \\
  複数行に渡って行頭に文字を挿入したいなど、通常の選択範囲では行えない編集を実現する。汎用性があり、複雑な編集も行うことができる。
\item \textgt{ウィンドウ分割}         \\
  同一、もしくは複数のファイルを同時に参照したいなどに活用される。また、巨大なディスプレイを利用している場合は表示領域を有効活用することにも有用である。
\item \textgt{履歴}                   \\
  編集履歴、クリップボード履歴、ファイルアクセス履歴、カーソル移動履歴など様々な履歴を扱うことができ「間違っても元に戻せる」という安心感を与える。
\item \textgt{バージョン管理}         \\
  開発の現場では、バージョン管理システムによってソースコードを管理する場合が多い。Emacs では標準で様々なバージョン管理システムに対応しており、Emacs から離れることなく透過的に利用することができる。
\item \textgt{辞書引き}               \\
  文字を入力する際、日本語・英語を問わず辞書を引きたくなることは頻繁にある。その度にブラウザや辞書アプリケーションを立ち上げるのは億劫であるが、Emacs では様々な辞書データを扱うための拡張機能が色々と用意されている。
\item \textgt{ドキュメント閲覧・検索} \\
  辞書と同じく、プログラムのマニュアル・ドキュメントを参照したいことはよくある。Emacs には UNIX マニュアルを読むための機能が用意されている。また、カーソル位置にある単語をドキュメントから検索するなど、エディタならではの使い方が実現できる。
\end{itemize}\vspc{-0.50zw}
これらの機能はそれぞれ独立しているが、組み合わせることで爆発的な効果が生まれる。
全て使いこなせるようになれば、エディタの価値観が大きく変わることになるだろう。
%%
%% 節:Emacs が Emacs であるための特徴
%%--------------------------------------------------------------------------------------------------------------------%%
\section{Emacs が Emacs であるための特徴}
前節でまとめたものは「エディタとして」の機能についてであったが、それら個別の機能とは別に「Emacs が Emacs であるための重要な特徴」がある。
本稿では、様々な機能を紹介しつつ Emacs の持つ次の 3 つの特徴をテーマに解説を進める。
%%
%% 項:優れた操作性
%%----------------------------------------------------------------------------------------------------------%%
\subsection{優れた操作性}
まず最初の特徴は、その優れた操作性である。
Emacs の操作は主に 2 つに分類される。\\

1 つ目はカーソル移動である。
ファイル内を縦横無尽にカーソル移動させ、コピーやカットなどのエディタらしい機能を使うための操作を指す。
これは小さな視点での操作とも言える。\\

2 つ目は Emacs 全体の操作で、大きな視点での操作と言える。
ファイルを開く、保存する、バッファ(これについては第 2 章で解説する)を切り替える、ウィンドウを分割するなど、全てキーボードから行うことができる。
勿論、マウスも利用可能だが、キーボードから手を離さず全てが操作可能ということは、非常に大きな意味を持つ。\\

更に、操作は次に挙げる設定の柔軟性によって自分好みにカスタマイズすることができる。
%%
%% 項:設定の柔軟性
%%----------------------------------------------------------------------------------------------------------%%
\subsection{設定の柔軟性}
Emacs と言えば、設定の柔軟性なくして語ることはできない。
前述した通り、Emacs のほとんどの機能が Elisp によって実装されている。
表示や操作はデフォルトの設定値が与えられているだけなので、その値を変更することによって、自分の好きなようにカスタマイズすることができるのである。\\

但し、Emacs は様々な機能が組み合わされて構成されているため、どこに記述された設定が反映されているのか、またその優先順位がどの様になっているのかを把握するためには少々経験が必要となる。
場合によっては思い通りにならないこともあるだろう。
しかし、そういったことも第 3 章で解説するヘルプ機能を利用して調べるスキルを身につけることによって解決することができるようになるだろう。
%%
%% 項:本体の拡張性
%%----------------------------------------------------------------------------------------------------------%%
\subsection{本体の拡張性}
最後に挙げるのは本体の拡張性である。
Emacs の機能は本体のバージョンが上がる度に増えている。
実は、それらの機能のほとんどは、誰かが Elisp を用いて作成したもので、評判の良いものをメーリングリストなどで議論して本体に組み込んでいるのである。\\

Emacs はフリーウェアの代表格とも言えるソフトウェアである。
非常に多くの開発者が開発を支えている。
しかし、何も本体の開発に直接参加している者達だけが貢献者ではない。
例えば、小さな機能を考えて Web 上に公開することも重要な素晴らしいコミットである。\\

自分が欲しいほんの小さな機能、そういったものも自分で作成可能ということは非常に大きな魅力ではないだろうか。
