%%
%% 章:インストール、設定ファイルと画面の構成
%%------------------------------------------------------------------------------------------------------------------------------%%
\chapter{インストール、設定ファイルと画面の構成}
%%
%% 節:インストール
%%--------------------------------------------------------------------------------------------------------------------%%
\section{インストール}
Emacs のインストールの仕方は幾つか存在するが、大きく分類すると次の 2 つに分けられる。
\vspc{-0.50zw}\begin{itemize}\setlength{\leftskip}{-1.00zw}%\setlength{\labelsep}{+1.00zw}
\item ファイルをダウンロードしてコピーするだけで利用可能なバイナリ\footnote{既にビルドされたソフトウェアのこと。}インストール。
\item ソースコードをダウンロードし、ビルドしてインストールするソースインストール。
\end{itemize}\vspc{-0.50zw}
本節では、そのどちらも解説いていく。
%%
%% 項:Mac へのインストール
%%----------------------------------------------------------------------------------------------------------%%
\subsection{Mac へのインストール}
Mac には予めターミナルで動作する Emacs~22.1 がインストールされている。
しかし、これは 2007 年にリリースされた古いバージョンであるため、本稿で紹介する機能を利用するためには最新バージョンをインストールする必要がある。\\

Mac 向けの Emacs には、ターミナル上で動作する Emacs 以外にもデスクトップ上で動作する Emacs.app が利用可能である。
後者は一般的な macOS アプリケーションと同じルック&フィールが用いられているため、普段使い慣れたアプリケーションと同じように Emacs を利用することができる。\\

Emacs.app のインストール方法は、主に既にビルドされた Emacs.app をダウンロードする方法とソースコードからインストールする 2 つの方法が存在する。
%%
%% 款:Emacs.app を入手する
%%------------------------------------------------------------------------------------------------%%
\subsubsection{Emacs.app を入手する}
macOS 向けの公式バイナリは存在しないが「\ruby{GNU}{グヌー} Emacs For Mac OS X」という準公式のバイナリがダウンロードできる Web サイト\footnote{http://emacsformacosx.com}が存在する。\\

そこにはリリース版、プレテスト版(リリース候補版)、ナイトリー版(毎日ビルドされる開発版)と様々なバイナリが用意されている。
ディスクイメージファイル(dmg ファイル)をダウンロードしてマウントし、開いたウィンドウ内に存在する Emacs のアイコンを Applications ディレクトリにドロップするだけでインストールすることが可能である。
%%
%% 款:Emacs.app を自分でビルドする
%%------------------------------------------------------------------------------------------------%%
\subsubsection{Emacs.app を自分でビルドする}
macOS 付属の開発環境である Command Line Tools がインストールされていれば、Emacs の本家である「GNU Emacs --- GNU Project」\footnote{http://www.gnu.org/software/emacs}からソースコードをダウンロードして Emacs.app を作成することもできる。\enlargethispage{1.00zw}
Terminal.app(ターミナル)などの端末から次のコマンドを実行することで Emacs.app を自分で簡単にビルドすることができる(\texttt{\$} はコマンドプロンプト)。
\vspc{+0.50zw}\begin{mdframed}[roundcorner=0.50zw,leftmargin=3.00zw,rightmargin=3.00zw,skipabove=0.40zw,skipbelow=0.40zw,innertopmargin=4.00pt,innerbottommargin=4.00pt,innerleftmargin=5.00pt,innerrightmargin=5.00pt,linecolor=gray!090,linewidth=0.50pt,backgroundcolor=gray!90]\color{gray!10}
\begin{verbatim}
$ curl -0 http://ftp.gnu.org/pub/gnu/emacs/emacs-25.2.tar.gz
$ tar xvf emacs-25.2.tar.gz
$ cd emacs-25.2
$ ./configure --with-ns
$ make install
\end{verbatim}
\end{mdframed}\vspc{-0.70zw}
これで「./emacs-25.2/nextstep」ディレクトリ内に Emacs.app が作成されるので、これを Applications ディレクトリにドラッグ&ドロップしてインストールする。\\

ビルドに関する詳細は「./emacs-25.2/nextstep/INSTALL」に説明があるので、必要に応じて参照するとよい。
%%
%% 項:Windows へのインストール
%%----------------------------------------------------------------------------------------------------------%%
\subsection{Windows へのインストール}
Windows 向けの Emacs は GNU Project が配布する Emacs、Windows Subsystem for Linux(以下、WSL と記す)のEmacs、そして Emacsen(Emacs 派生エディタ)である Meadow など、様々なディストリビューションが存在し、情報が少々錯綜している。
特に問題がなければ、公式に配布されている Emacs  を利用するとよいだろう。\\

ただ、Windows 10 から利用可能となった WSL では Linux と同じ環境が実現できるため、grep などの Linux コマンドを Emacs と組み合わせて利用したい者にとっては有用な選択肢となる。
そのため、WSL を扱える場合はこちらの Emacs を利用してみるのもよいだろう。
%%
%% 款:オフィシャルビルドを利用する
%%------------------------------------------------------------------------------------------------%%
\subsubsection{オフィシャルビルドを利用する}
Windows に Emacs をインストールするには、バイナリをダウンロードしてインストールするのが一番簡単である。
オフィシャルビルドは次のアドレスからダウンロードすることができる。
\begin{quote}
  https://ftp.gnu.org/gnu/emacs/windows/emacs-27/emacs-27.1-x86\_64.zip
\end{quote}
解凍したフォルダを任意の場所(C:\textyen{}emacs など)にコピーしてインストールする。
%%
%% 項:Linux へのインストール
%%----------------------------------------------------------------------------------------------------------%%
\subsection{Linux へのインストール}
Linux では既に Emacs がインストールされている場合が殆どだが、ディストリビューションによってはインストールされていなかったり、バージョンが最新ではない場合がある。
もし、最新バージョンのパッケージが用意されているようなら、パッケージマネージャーを利用してインストールするのが手軽である。\\

パッケージに最新の Emacs が存在しない場合は、次のコマンドを利用してインストールする。
但し、X Window System(以下、Xと記す)を使用せず、ターミナル環境のみで Emacs を利用する場合は、次の項で解説する方法でインストールする必要がある。
\vspc{+0.50zw}\begin{mdframed}[roundcorner=0.50zw,leftmargin=3.00zw,rightmargin=3.00zw,skipabove=0.40zw,skipbelow=0.40zw,innertopmargin=4.00pt,innerbottommargin=4.00pt,innerleftmargin=5.00pt,innerrightmargin=5.00pt,linecolor=gray!090,linewidth=0.50pt,backgroundcolor=gray!90]\color{gray!10}
\begin{verbatim}
$ curl -0 http://ftp.gnu.org/pub/gnu/emacs/emacs-27.1.tar.gz
$ tar xvf emacs-27.1.tar.gz
$ cd emacs-27.1
$ ./configure
$ make
$ sudo make install
\end{verbatim}
\end{mdframed}\vspc{-1.70zw}
%%
%% 項:ターミナル環境へのインストール
%%----------------------------------------------------------------------------------------------------------%%
\subsection{ターミナル環境へのインストール}
ターミナル環境への Emacs をビルドする場合、次のような方法が最もシンプルである。
このインストール方法は Mac と Linux で共通である。\enlargethispage{0.50zw}
\vspc{+0.50zw}\begin{mdframed}[roundcorner=0.50zw,leftmargin=3.00zw,rightmargin=3.00zw,skipabove=0.40zw,skipbelow=0.40zw,innertopmargin=4.00pt,innerbottommargin=4.00pt,innerleftmargin=5.00pt,innerrightmargin=5.00pt,linecolor=gray!090,linewidth=0.50pt,backgroundcolor=gray!90]\color{gray!10}
\begin{verbatim}
$ curl -0 http://ftp.gnu.org/pub/gnu/emacs/emacs-27.1.tar.gz
$ tar xvf emacs-27.1.tar.gz
$ cd emacs-27.1
$ ./configure --without-x
$ make
$ sudo make install
\end{verbatim}
\end{mdframed}\vspc{-0.70zw}
これで「/usr/local/」へ Emacs がインストールされる。\\

起動方法については第 3 章で解説する。
%%
%% 節:ディレクトリと設定ファイルの構成
%%--------------------------------------------------------------------------------------------------------------------%%
\section{ディレクトリと設定ファイルの構成}
Emacs はベースとなる部分以外の機能は Elisp によって実装されている。
そのため、Emacs をインストールすると大量の Elisp も一緒にインストールされる。
OS によって多少インストールされる場所は異なるが、ディレクトリ構造はほぼ共通である。
%%
%% 項:各ディレクトリの役割
%%----------------------------------------------------------------------------------------------------------%%
\subsection{各ディレクトリの役割}
Emacs 本体と一緒にインストールされるディレクトリの中で、重要なものを紹介しておく。
%%
%% 款:etc ディレクトリ
%%------------------------------------------------------------------------------------------------%%
\subsubsection{etc ディレクトリ}
etc ディレクトリには Emacs の NEWS やライセンスなどのドキュメントが格納されている。
NEWS には追加された機能などの情報が掲載されている。
%%
%% 款:leim ディレクトリ
%%------------------------------------------------------------------------------------------------%%
\subsubsection{leim ディレクトリ}
leim ディレクトリには Emacs 標準の IM(Input Method)が格納されている。
特に設定しなくても利用することができるのは魅力的だが、IM としてはあまり強力ではない。
Emacs で快適な日本語入力を求めるのであれば、SKK(Simple Kana Kanji conversion program)を推奨する。
%%
%% 款:lisp ディレクトリ
%%------------------------------------------------------------------------------------------------%%
\subsubsection{lisp ディレクトリ}
lisp ディレクトリには Emacs 同梱の Elisp が格納されている。
Emacs の全てがこのディレクトリに詰まっていると言っても過言ではないだろう。
%%
%% 款:site-lisp ディレクトリ
%%------------------------------------------------------------------------------------------------%%
\subsubsection{site-lisp ディレクトリ}
site-lisp ディレクトリは、ユーザが Elisp をインストールするために標準で用意されているディレクトリである。
ロードパス(Emacs が Elisp を探すディレクトリ)が通っているため、このディレクトリにインストールされた Elisp は load や require などの関数を利用することで読み込むことができる。\\

しかし、複数ユーザで同じ Elisp を利用する目的以外では site-lisp に Elisp をインストールすることは推奨しない。
.emacs.d ディレクトリの中にインストールする方法を利用すると.emacs.d ディレクトリをコピーするだけで環境がコピーできるため、そちらを推奨する。
詳しくは、第 4 章「\textasciitilde{}/.emacs.d ディレクトリに設定をまとめて管理」で解説する。
%%
%% 款:bin ディレクトリ
%%------------------------------------------------------------------------------------------------%%
\subsubsection{bin ディレクトリ}
bin ディレクトリは Windows 版の Emacs のみに存在し、Emacs 本体や Emacs から利用するためのコマンドの実行ファイルが格納されている。
また、このディレクトリに配置された実行ファイルは Emacs から呼び出し可能となる。
%%
%% 款:ホームディレクトリに作成される .emacs.d ディレクトリ
%%------------------------------------------------------------------------------------------------%%
\subsubsection{ホームディレクトリに作成される.emacs.d ディレクトリ}
Emacs を起動すると、ホームディレクトリに.emacs.d というディレクトリが作成される。
このディレクトリは Emacs の設定ファイルを配置するために用いる。
%%
%% 項:Windows のホームディレクトリ設定
%%----------------------------------------------------------------------------------------------------------%%
\subsection{Windows のホームディレクトリ設定}
Windows 10 の初期状態では「C:{\textyen}Users{\textyen}ユーザ名{\textyen}AppDate{\textyen}Roaming」をホームディレクトリとして利用する。
このディレクトリは Windows の環境変数 HOME を設定することで自由に変更することができる。\\

Windows 10 の場合[コントロールパネル]の[システムとセキュリティ]から[システム]の[システムの詳細設定]を開いて[詳細設定]タブの[環境変数]をクリックし[ユーザー環境変数]に、次のように環境変数を新規追加する。
\vspc{-0.50zw}\begin{itemize}\setlength{\leftskip}{-1.00zw}%\setlength{\labelsep}{+1.00zw}
\item 変数名:HOME
\item 変数値:値は任意。設定を作成したいフォルダを指定する。例:C:{\textyen}Users{\textyen}ユーザ名
\end{itemize}\vspc{-0.50zw}
この設定は次回ログイン後に反映され、Emacs を起動すると設定したフォルダに.emacs.d ディレクトリが自動的に生成される。
%%
%% 項:設定ファイルの構成
%%----------------------------------------------------------------------------------------------------------%%
\subsection{設定ファイルの構成}
Emacs の初期化ファイル(設定ファイル)は.emacs と呼ばれる。
ドットファイルというのは UNIX 系 OS 特有の設定ファイルのことで、ドットから名前が始まることに由来する。
ドットファイルは通常隠しファイルとなっており、意図せず削除してしまったりしないようになっている。\\

第 1 章において Emacs の特徴として「設定の柔軟性」を挙げたが、その柔軟な設定は設定ファイルに記述することで Emacs を起動する際に一度だけ読み込まれ反映される。
設定ファイにルは.emacs や init.el など複数の名前が用意されているが、読み込まれるのは 1 つだけである。
その優先順位は次のようになっている。
\vspc{-0.50zw}\begin{itemize}\setlength{\leftskip}{+0.00zw}%\setlength{\labelsep}{+1.00zw}
\item[\ajMaru{1}] \textasciitilde/.emacs.el(.emacs.elc)
\item[\ajMaru{2}] \textasciitilde/.emacs
\item[\ajMaru{3}] \textasciitilde/.emacs.d/init.el(init.elc)
\item[\ajMaru{4}] \textasciitilde/.emacs.d/init
\end{itemize}\vspc{-0.50zw}
全てのファイルはバイトコンパイル(後述)することで elc ファイルに変換することができる。
Emacs では el ファイルと同名の elc ファイルが存在する場合は elc ファイルを読み込み、el ファイルに記述された設定は一切読み込まれないことに注意が必要である。\\

上記のどのファイルに設定を記述しても Emacs の動作に違いはない。
もし、どのファイルに記述すればよいか迷った場合は init.el を推奨する。
なぜなら.emacs.d ディレクトリをバックアップするだけで Emacs の設定をバックアップすることができるためである。
詳しくは第 4 章で解説する。
%%
%% 節:画面の構成
%%--------------------------------------------------------------------------------------------------------------------%%
\section{画面の構成}
Emacs の画面について解説していく。
%%
%% 項:フレーム
%%----------------------------------------------------------------------------------------------------------%%
\subsection{フレーム}
Windows や Mac でウィンドウと呼ばれるアプリケーションの表示枠のことを、Emacs ではフレーム(\emph{frame})と呼ぶ。
frame と名の付くコマンドや変数(後述)は、このフレームに関する操作を行うものである。
%%
%% 項:ウィンドウ
%%----------------------------------------------------------------------------------------------------------%%
\subsection{ウィンドウ}
Emacs のフレーム上で、ファイル(正確には後述するバッファ)を表示している領域のことをウィンドウと呼ぶ。
ウィンドウは分割することが可能で、フレーム内に幾つものウィンドウを並べることができる。
また、Emacs は複数のファイルを同時に開くことができるため、異なるファイルや同じファイルの異なる箇所を別のウィンドウに表示しながら編集することができる。
%%
%% 項:フリンジ
%%----------------------------------------------------------------------------------------------------------%%
\subsection{フリンジ}
ウィンドウの両端に位置するアイコンを表示可能な細い特殊なウィンドウで、主に行に関する追加情報を視覚的に表示するための場所をフリンジと呼ぶ。
例えば、実際には改行されていないが Emacs の画面上では行の折り返しが行われている場合は、折り返し記号を表示する。
他にも、EOF(End Of File)などを表示させることも可能である。
%%
%% 項:バッファ
%%----------------------------------------------------------------------------------------------------------%%
\subsection{バッファ}
一般的なコンピュータ用語では、情報を一時的に蓄える記憶領域のことをバッファと呼ぶが、Emacs ではメモリ上に作成されたオブジェクトのことをバッファと呼ぶ。
更に直感的に説明するなら、ウィンドウの中に表示されているもの全てがバッファである。
Emacs でファイルを開いた場合、そのファイルの内容が全てバッファへと読み込まれてウィンドウに表示される。
すなわち、ファイルは Emacs 上では全てバッファと呼ばれることになる。\\

バッファはあくまでも Emacs 上のオブジェクトなので、ファイルとして保存しない限りその編集はファイルに反映されることはない。
また、一度作成されたバッファは明示的に消去しない限り(または Emacs を終了しない限り)Emacs 上に残り続け、バッファを消去することによってメモリから開放される。
そして、Elisp によるプログラミングではバッファを変数のように扱うこともできる。
%%
%% 項:モードライン
%%----------------------------------------------------------------------------------------------------------%%
\subsection{モードライン}
モードラインは、ある意味究極的に素っ気ない Emacs において唯一華のある部分である。
ウィンドウの下部に存在し、カレントバッファ(編集中のバッファ)に関する様々な情報を表示する。
また、バッファに関係のない情報であっても Emacs からアクセス可能であれば何でもモードラインに表示することができる。
%%
%% 項:ミニバッファ(エコーエリア)
%%----------------------------------------------------------------------------------------------------------%%
\subsection{ミニバッファ(エコーエリア)}
Emacs のフレーム最下部に存在するミニバッファは、Emacs へのコマンドや引数を入力する場所として利用する。
ミニバッファでは \raise0.3ex\hbox{\ovalbox{\footnotesize{\texttt{tab\vphantom{p}}}}} や \raise0.3ex\hbox{\ovalbox{\footnotesize{\texttt{space\vphantom{b}}}}} による補完機能が利用できることが多い。
また、この場所は Emacs からのメッセージを表示するエコーエリアとしても利用される。
同じ場所を使用しているが、内部的にはミニバッファとエコーエリアは全くの別物である。
%%
%% 節:モード
%%--------------------------------------------------------------------------------------------------------------------%%
\section{モード}
Emacs を操作する上で必ず理解すべきものがモードという概念である。
モードとは一体どういうもので、何を提供してくれるのかを知ることで、Emacs に対する理解が深まるだろう。
%%
%% 項:メジャーモード
%%----------------------------------------------------------------------------------------------------------%%
\subsection{メジャーモード}
メジャーモードとは、バッファに対して「必ず 1 つ」適用されるモードである。
例えば、*scratch* バッファでは lisp-interaction-mode という Lisp システム実行環境用のモードが適用され、rb ファイルを開くとプログラミング言語 Ruby を編集するための ruby-mode が適用される。
現在のメジャーモードはモードラインに表示される。
%%
%% 款:メジャーモードを選択する仕組み
%%------------------------------------------------------------------------------------------------%%
\subsubsection{メジャーモードを選択する仕組み}
Emacs が自動的にメジャーモードを選択する仕組みは非常に単純で、ファイル名(主に拡張子)とファイルの shebang(シバン)を元にモードを決定する。\\

auto-mode-alist はファイル名からモードを選択する仕組みである。
この変数にはファイル名にマッチさせるための正規表現とマッチした際に選択されるモードのリストが記述されており、先頭から順番にチェックしてマッチした瞬間にそのモードを選択してファイルを開く。\\

interpreter-mode-alist は shebang からモードを選択する仕組みである。
この変数にはファイルの最初の行にある shebang に記述されているインタプリタ名にマッチさせるための正規表現と、マッチした際に選択されるモードのリストが記述されている。
Emacs は shebang が存在する場合に限って、このリストを利用してモードを選択する。
尚、優先順位は interpreter-mode-alist $>$ auto-mode-alist となっている。
また、もし何もマッチしなければデフォルトで text-mode というテキストファイルを編集するためのメジャーモードが選択される。
%%
%% 款:メジャーモードが提供する機能
%%------------------------------------------------------------------------------------------------%%
\subsubsection{メジャーモードが提供する機能}
メジャーモードが提供する機能には、主に次のようなものがある。
\vspc{-0.50zw}\begin{itemize}\setlength{\leftskip}{-1.00zw}%\setlength{\labelsep}{+1.00zw}
\item メジャーモード専用コマンド(例えば、lisp-interaction-mode で式を評価し戻り値を出力する eval-print-last-sexp など)
\item 特別なキーバインド(キーマップ)
\item シンタックスハイライト
\item フック(第 5 章で後述)
\end{itemize}\vspc{-0.50zw}
%%
%% 項:マイナーモード
%%----------------------------------------------------------------------------------------------------------%%
\subsection{マイナーモード}
マイナーモードはメジャーモードと異なり、 1 つのバッファに対して複数適用することができる。
その最大の特徴は、使いたい際に有効化して必要がなければ無効化することができる点である。
マイナーモードが提供する機能は、補完を手助けしてくれたり、表示を大きくしたり実に様々である。
現在のマイナーモードもモードラインに表示される。
%%
%% 款:マイナーモードが提供する機能
%%------------------------------------------------------------------------------------------------%%
\subsubsection{マイナーモードが提供する機能}
マイナーモードが提供する機能には、主に次のようなものがある。
\vspc{-0.50zw}\begin{itemize}\setlength{\leftskip}{-1.00zw}%\setlength{\labelsep}{+1.00zw}
\item マイナーモード専用コマンド(例えば、auto-complete-mode で自動補完を行う auto-complete など)
\item 特別なキーバインド(キーマップ)
\item フック(フックが定義されている場合)
\end{itemize}\vspc{-0.50zw}
マイナーモードについて詳しくは第 7 章で解説する。
