%%
%% プリアンブル
%%==============================================================================================================================%%
\RequirePackage{fix-cm}
%%
%% ドキュメントクラスとオプションの指定
%%--------------------------------------------------------------------------------------------------------------------%%
\documentclass[10pt,a4paper,disablejfam,dvipdfmx,fleqn,onecolumn,oneside,openany,report]{jsarticle}
%%
%% パッケージの読み込み
%%--------------------------------------------------------------------------------------------------------------------%%
\input{/usr/local/etc/Package}
%%
%% コマンドと環境の定義
%%--------------------------------------------------------------------------------------------------------------------%%
\input{/usr/local/etc/Defines}
%%
%% ページレイアウト設定(A4横組用レイアウト for jsreport)
%%--------------------------------------------------------------------------------------------------------------------%%
\input{/usr/local/etc/Layouts}
%%
%% タイトル・著者名・作成日の設定
%%--------------------------------------------------------------------------------------------------------------------%%
\title{質問回答} \author{姫 伯邑考} \date{2017年01月24日}
%%
%% 本文
%%====================================================================================================================%%
\begin{document}
\maketitle
\begin{itemize}
\item[\ajMaru{15}] $a$ を定数とする。2次方程式 $x^{2}+5x-(a^{2}+a-7) = 0$ が異なる2つの解をもつとき $a$ の値の範囲は \ovalbox{ア} である。
  このとき、2次方程式 $x^{2}+5x-(a^{2}+a-7) = 0$ の2解の差の絶対値は \ovalbox{イ} である。また、2次不等式 $x^{2}+5x \le a^{2}+a-7$ を満たす整数 $x$ が存在しないとき $a$ の値の範囲は \ovalbox{ウ} である。
  \begin{solve}
    \begin{itemize}
    \item[ア)] 2次方程式 $x^{2}+5x-(a^{2}+a-7)=0$ の解の判別式を $D$ をすると、この2次方程式が異なる2つの実数解をもつための条件は次式で与えられる。
      \begin{align*}
        D = 5^{2}+4(a^{2}+a-7) > 0
      \end{align*}
      この2次不等式を解くと、
      \begin{align*}
        5^{2}+4(a^{2}+a-7) > 0 \hspc{5.00pt} \Leftrightarrow \hspc{5.00pt} 4a^{2}+4a-3 > 0 \hspc{5.00pt} \Leftrightarrow \hspc{5.00pt} (2a+3)(2a-1) > 0 \hspc{5.00pt} \Leftrightarrow \hspc{5.00pt} a <-\frac{3}{2},\ \frac{1}{2} < a
      \end{align*}
      故に、$a<-\frac{3}{2}$、$\frac{1}{2}<a$ である。$\cdots$ \ovalbox{ア}

    \item[イ)] 2次方程式の2解を $\alpha$、$\beta$ とおくと、解と係数の関係より、
      \begin{align*}
        \alpha + \beta =-5,\hspc{+10pt} \alpha\beta =-(a^{2}+a-7)
      \end{align*}
      ここで、
      \begin{align*}
        (\alpha-\beta)^{2} & = (\alpha+\beta)^{2}-4\alpha\beta \\ & = 25+4(a^{2}+a-7) = 4a^{2}+4a-3 \hspc{5.00pt} \Leftrightarrow \hspc{5.00pt} |\alpha-\beta| = \sqrt{4a^{2}+4a-3}
      \end{align*}
      故に、$\sqrt{4a^{2}+4a-3}$ である。$\cdots$ \ovalbox{イ}
    \end{itemize}

  \item[ウ)] $k = x^{2}+5x$ とおくと、$k$ の最小値は、
    \begin{align*}
      k = x^{2}+5x = \left(x+\frac{5}{2}\right)^{2}-\frac{25}{4}
    \end{align*}
    より、$x=-\frac{5}{2}$ のときの値 $-\frac{25}{4}(=-6.25)$ である。\\
    従って、$a^{2}+a-7$ の値が $-6$ より小さければ、与式は整数解をもたない。故に、
    \begin{align*}
      & a^{2}+a-7 <-6 \\
      & \Leftrightarrow \hspc{5.00pt} a^{2}+a-1 < 0 \hspc{5.00pt} \Leftrightarrow \hspc{5.00pt} \left(a-\frac{-1-\sqrt{5}}{2}\right)\left(a-\frac{-1+\sqrt{5}}{2}\right) < 0 \hspc{5.00pt} \Leftrightarrow \hspc{5.00pt} \frac{-1-\sqrt{5}}{2} < a < \frac{-1+\sqrt{5}}{2}
    \end{align*}
    以上より、$\frac{-1-\sqrt{5}}{2} < a < \frac{-1+\sqrt{5}}{2}$ である。
  \end{solve}
\end{itemize}
\end{document}
