%%
%% プリアンブル
%%==============================================================================================================================%%
\RequirePackage{fix-cm}
%%
%% ドキュメントクラスとオプションの指定
%%--------------------------------------------------------------------------------------------------------------------%%
\documentclass[10pt,a4paper,disablejfam,dvipdfmx,fleqn,onecolumn,oneside,openany,report]{jsarticle}
%%
%% パッケージの読み込み
%%--------------------------------------------------------------------------------------------------------------------%%
\input{/usr/local/etc/Package}
%%
%% コマンドと環境の定義
%%--------------------------------------------------------------------------------------------------------------------%%
\input{/usr/local/etc/Defines}
%%
%% ページレイアウト設定(A4横組用レイアウト for jsreport)
%%--------------------------------------------------------------------------------------------------------------------%%
\input{/usr/local/etc/Layouts}
%%
%% タイトル・著者名・作成日の設定
%%--------------------------------------------------------------------------------------------------------------------%%
\title{質問回答} \author{姫 伯邑考} \date{2020年09月22日}
%%
%% 本文
%%====================================================================================================================%%
\begin{document}
\maketitle
\begin{itemize}
\item[235] 次の不等式を解きなさい。
  \vspc{-0.00zw}\begin{itemize}\setlength{\leftskip}{-1.00zw}\vspc{+0.50zw}%\setlength{\labelsep}{+1.00zw}
  \item[(1)] $\displaystyle{7-\frac{5(1-x)}{3}} \ge 2(x-1)$ \vspc{+0.50zw}
    \begin{solve}
      与式の両辺を 3 倍して式を整理すると、
      \begin{align*}
        21-5(1-x) \ge 6(x-1) \hspc{+5.00pt} \Rightarrow \hspc{+5.00pt} 21+5(x-1) \ge 6(x-1) \hspc{+5.00pt} \Rightarrow \hspc{+5.00pt} x-1 \le 21 \hspc{+5.00pt} \Rightarrow \hspc{+5.00pt} x \le 22
      \end{align*}
      注)$-5(1-x) = 5(x-1)$ \\
    \end{solve}
  \end{itemize}\vspc{-0.50zw}
\end{itemize}
\end{document}
