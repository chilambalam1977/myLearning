%%
%% 章:数と式
%%------------------------------------------------------------------------------------------------------------------------------%%
\chapter{数と式}
%%
%% 節:整式の加法・減法・乗法
%%--------------------------------------------------------------------------------------------------------------------%%
\section{整式の加法・減法・乗法}
%%
%% 款:単項式とその係数・次数
%%------------------------------------------------------------------------------------------------%%
\subsubsection{単項式とその係数・次数}
数や文字、及びそれらをかけ合わせてできる式を\textcolor{red}{単項式}という。
単項式において、数の部分をその単項式の\textcolor{red}{係数}といい、掛け合わせされた文字の個数をその単項式の\textcolor{red}{次数}という。\\

2 種類以上の文字を含む単項式において、特定の文字に着目して係数や次数を考えることがある。
この場合、着目した文字以外の文字は数と同様、すなわち係数として扱う。
